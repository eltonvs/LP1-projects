\documentclass[12pt,openright,oneside,a4paper,english,brazil]{abntex2}

\usepackage[brazil]{babel}
\usepackage[utf8]{inputenc}
\usepackage{indentfirst}

\autor{Elton de Souza Vieira}
\titulo{Relatório-Resumo sobre Tabelas de Dispersão}
\data{2016}
\local{Natal - RN}
\tipotrabalho{relatorio}

\DoubleSpacing

\begin{document}
\imprimircapa

Basicamente, uma Tabela de Dispersão é uma estrutura de dados do tipo dicionário, ou seja, associa uma chave a um valor. Ela não armazena elementos repetidos, nem estabelece uma ordem entre eles (de forma que possa acessá-los sequencialmente) e, por causa disso, ela não consegue recuperar o sucessor ou antecessor a um elemento.

\end{document}